%! Author = zjs
%! Date = 6/27/22

% Preamble
\documentclass{ctexbeamer}
% Packages
\usepackage{braket}
\usepackage{fancyvrb}
\usepackage{textcomp}
\usepackage{amsmath}
\usepackage{xcolor}
\usepackage{tcolorbox}
\usepackage{siunitx}

\newcommand{\ul}[1]{\underline{#1}}
\newcommand{\red}[1]{\textcolor{red}{#1}}
\newcommand{\pink}[1]{\textcolor{pink}{#1}}
\newcommand{\green}[1]{\textcolor{green}{#1}}
\newcommand{\gray}[1]{\textcolor{gray}{#1}}
\newcommand{\lightgray}[1]{\textcolor{lightgray}{#1}}
\newcommand{\lime}[1]{\textcolor{lime}{#1}}
\newcommand{\purple}[1]{\textcolor{purple}{#1}}
\newcommand{\str}[1]{\texttt{#1}}
\newcommand{\fail}{\mathsf{fail}}
\newcommand{\ugrn}[1]{\underline{\green{#1}}}
\newcommand{\ured}[1]{\underline{\red{#1}}}
\DefineVerbatimEnvironment%tight verbatim
{tverb}{Verbatim} %在tverb环境里,verbamtim 的 \, {, } 需要escape。
{commandchars=\\\{\},baselinestretch=0}


\DefineVerbatimEnvironment%tight verbatim
{bverb}{BVerbatim}
{commandchars=\\\{\},baselinestretch=0, fontsize=\small}

\DefineVerbatimEnvironment%tight verbatim for code
{cverb}{BVerbatim}
{baselinestretch=0, fontsize=\tiny}


% Document
\begin{document}
\title{KMP算法}
\maketitle

\begin{frame}
  \frametitle{约定、记号、概念}

  \begin{itemize}
    \item 字符串下标从1开始。
  \end{itemize}


$S$ 是一个字符串。
\begin{itemize}
\item $|S|$:$S$ 的长度。
\item  $S_i$:$S$ 的第 $i$ 个字符。$1 \le i \le |S|$。
\item $S_{l..r}$:$S$ 的第 $l$ 到第 $r$ 个字符构成的子串。
\item $S_{1..i}$ 称为 $S$ 的前缀。
\item $S_{i..|S|}$ 称为 $S$ 的后缀。
\item 空串是每个字符串的前缀和后缀。
\end{itemize}

\end{frame}
\begin{frame}[fragile]
	\frametitle{问题:字符串匹配}

  

\begin{tcolorbox}
  模式串$P$,文本串$T$。
  模式串在文本串的哪些地方出现过?
  \tcblower
  例子:$P=\str{aba}$,$T=\str{ababa}$。

  答案:$\set{1, 3}$。
\end{tcolorbox}


\begin{itemize}
\item 文本串长度记作 $n$
\item 模式串长度记作 $m$
\end{itemize}

若 $T_{i..m+i-1} = P$ 则称 $i$ 是\emph{匹配点}。

\end{frame}


\begin{frame}[fragile]
\frametitle{暴力匹配算法}

\begin{itemize}
\item 第$i$次匹配:把模式串开头和文本串第$i$个字符对齐,从头开始比较。遇到文本串和模式串里对应字符不相同(称为失配)就停止。
\item 共匹配 $n-m+1$次。
\end{itemize}


\begin{tcolorbox}
\begin{cverb}
for (int i = 1; i <= n - m + 1; i++) {
  int j;
  for (j = 1; j <= m; j++)
    if (T[i + j - 1] != P[j]) break;
  if (j > m)
    i是匹配点。
}
\end{cverb}
\end{tcolorbox}


\end{frame}

\begin{frame}[fragile]
\frametitle{暴力匹配算法示例}
P= \texttt{abcabcacab},T=\texttt{babcbabcabcaabcabcabcacabc}。

\begin{minipage}[t]{0.5\textwidth}
\begin{tverb}[frame=bottomline, fontsize=\small]
abcabcacab \(1\)
\red{b}abcbabcabcaabcabcabcacabc
\end{tverb}

\begin{tverb}[frame=bottomline, fontsize=\small]
 abcabcacab \(4\)
b\green{abc}\red{b}abcabcaabcabcabcacabc
\end{tverb}


\begin{tverb}[frame=bottomline, fontsize=\small]
  abcabcacab \(1\)
ba\red{b}cbabcabcaabcabcabcacabc
\end{tverb}

\begin{tverb}[frame=bottomline, fontsize=\small]
   abcabcacab \(1\)
bab\red{c}babcabcaabcabcabcacabc
\end{tverb}

\begin{tverb}[frame=bottomline, fontsize=\small]
    abcabcacab \(1\)
babc\red{b}abcabcaabcabcabcacabc
\end{tverb}

\begin{tverb}[frame=bottomline, fontsize=\small]
     abcabcacab \(8\)
babcb\green{abcabca}\red{a}bcabcabcacabc
\end{tverb}

\begin{tverb}[frame=bottomline, fontsize=\small]
      abcabcacab \(1\)
babcba\red{b}cabcaabcabcabcacabc
\end{tverb}

\begin{tverb}[frame=bottomline, fontsize=\small]
       abcabcacab \(1\)
babcbab\red{c}abcaabcabcabcacabc
\end{tverb}

\begin{tverb}[frame=bottomline, fontsize=\small]
        abcabcacab \(5\)
babcbabc\green{abca}\red{a}bcabcabcacabc
\end{tverb}

\end{minipage}%
\begin{minipage}[t]{.5\textwidth}

\begin{tverb}[frame=bottomline, fontsize=\small]
         abcabcacab \(1\)
babcbabca\red{b}caabcabcabcacabc
	\end{tverb}

\begin{tverb}[frame=bottomline, fontsize=\small]
          abcabcacab \(1\)
babcbabcab\red{c}aabcabcabcacabc
\end{tverb}

\begin{tverb}[frame=bottomline, fontsize=\small]
           abcabcacab \(2\)
babcbabcabc\green{a}\red{a}bcabcabcacabc
    	\end{tverb}

\begin{tverb}[frame=bottomline, fontsize=\small]
            abcabcacab \(7\)
babcbabcabca\green{abcabca}\red{b}cacabc
            	\end{tverb}

\begin{tverb}[frame=bottomline, fontsize=\small]
             abcabcacab \(1\)
babcbabcabcaa\red{b}cabcabcacabc
\end{tverb}

\begin{tverb}[frame=bottomline, fontsize=\small]
              abcabcacab \(1\)
babcbabcabcaab\red{c}abcabcacabc
            	\end{tverb}

\begin{tverb}[frame=bottomline, fontsize=\small]
               abcabcacab \(10\)
babcbabcabcaabc\green{abcabcacab}c
\end{tverb}

\begin{tverb}[frame=bottomline, fontsize=\small]
                abcabcacab \(1\)
babcbabcabcaabca\red{b}cabcacabc
\end{tverb}


\end{minipage}



一共比较了 $47$ 次。

\end{frame}

\begin{frame}
	\frametitle{暴力匹配算法的时间复杂度}

最坏情况时间复杂度是 $O(nm)$。


P=\texttt{aaaaaaab},T=\texttt{aaaaaaaaaaaaaaaaaaaab}。
\end{frame}

\begin{frame}[fragile]
\frametitle{改进暴力匹配算法}

观察暴力匹配算法的前五次匹配:

	\begin{tverb}[frame=bottomline, fontsize=\small]
abcabcacab \(1\)
\red{b}abcbabcabcaabcabcabcacabc
	\end{tverb}

\begin{tverb}[frame=bottomline, fontsize=\small]
 abcabcacab \(4\)
b\green{abc}\red{b}abcabcaabcabcabcacabc
	\end{tverb}


	\begin{tverb}[frame=bottomline, fontsize=\small]
  abcabcacab \(1\)
ba\red{b}cbabcabcaabcabcabcacabc
	\end{tverb}

	\begin{tverb}[frame=bottomline, fontsize=\small]
   abcabcacab \(1\)
bab\red{c}babcabcaabcabcabcacabc
	\end{tverb}

	\begin{tverb}[frame=bottomline, fontsize=\small]
    abcabcacab \(1\)
babc\red{b}abcabcaabcabcabcacabc
	\end{tverb}

从``第二次失配发生在$P_4, T_5$处'',我们可以知道关于文本串$T$的两个信息:
\begin{itemize}
\item $T_{2..4} = P_{1..3}$
\item $T_5 \ne P_4$
\end{itemize}
用这两个条件就能断定随后的三次暴力匹配在$P_1$处就失配。 
\begin{itemize}
  \item $T_2 = P_2; P_2 \ne P_1 \implies T_2 \ne P_1$
  \item $T_3 = P_3; P_3 \ne P_1 \implies T_3 \ne P_1$
  \item $T_5 \ne P_4; P_4 = P_1 \implies T_4 \ne P_1$
\end{itemize}

\end{frame}

\begin{frame}
  \frametitle{改进暴力匹配算法}
从上面的分析可以看出,如果我们{\kaishu 事先好好地研究模式串 $P$,把它的一切性质都记住},那么在暴力比配的过程中,一旦我们获得文本串 $T$ 的一些信息,以后就能避免将 $P$ 里的字符和 $B$ 里的字符进行{\kaishu 不必要的比较}。 

\tcbox{充分利用模式串就能避免不必要的比较。}

当我们获得下面两条信息以后
  \begin{itemize}
    \item $T_{2..4} = P_{1..3}$
    \item $T_5 \ne P_4$
  \end{itemize}
可以(也应当)做到
\begin{itemize}
  \item 不必再将 $P$ 里的字符和 $T_{2}$,$T_{3}$, $T_{4}$ 做比较。
  \item 不必再将 $P$ 里等于 $P_4$ 的字符和 $T_5$ 做比较。
\end{itemize}
\end{frame}

\begin{frame}[fragile]
\frametitle{改进暴力匹配算法}

\begin{tverb}[frame=bottomline, fontsize=\small]
abcabcacab
\red{b}abcbabcabcaabcabcabcacabc \(1\)
\end{tverb}

	\begin{tverb}[frame=bottomline, fontsize=\small]
 abcabcacab
\gray{b}\green{abc}\red{b}abcabcaabcabcabcacabc \(4\)
	\end{tverb}

\begin{tverb}[frame=bottomline, fontsize=\small]
     abcabcacab
\gray{ba}\pink{bcb}abcabcaabcabcabcacabc
\end{tverb}

\begin{tverb}[frame=bottomline, fontsize=\small]
     abcabcacab
\gray{babcb}\green{abcabca}\red{a}bcabcabcacabc \(8\)
\end{tverb}

\begin{tverb}[frame=bottomline, fontsize=\small]
        abcabcacab
\gray{babcba}\pink{bc}\lime{abca}\purple{a}bcabcabcacabc
\end{tverb}

\begin{tverb}[frame=bottomline, fontsize=\small]
        abcabcacab
\gray{babcbabc}\lime{abca}\red{a}bcabcabcacabc \(1\)
\end{tverb}

\begin{tverb}[frame=bottomline, fontsize=\small]
            abcabcacab
\gray{babcbabca}\pink{bca}\purple{a}bcabcabcacabc
\end{tverb}

\begin{tverb}[frame=bottomline, fontsize=\small]
            abcabcacab
\gray{babcbabcabca}\green{abcabca}\red{b}cacabc \(8\)
\end{tverb}

\begin{tverb}[frame=bottomline, fontsize=\small]
               abcabcacab
\gray{babcbabcabcaa}\pink{bc}\lime{abca}\purple{b}cacabc
\end{tverb}

\begin{tverb}[frame=bottomline, fontsize=\small]
               abcabcacab
\gray{babcbabcabcaabc}\lime{abca}\green{bcacab}c \(6\)
\end{tverb}

\begin{tverb}[frame=bottomline, fontsize=\small]
                       abcabcacab
\gray{babcbabcabcaabca}\pink{bcabcac}\lime{ab}c           一共比较了\(28\)次
\end{tverb}

\end{frame}

\begin{frame}
\frametitle{改进暴力匹配算法}


当 $T_i$ 和 $P_j$ 失配时(此时 $P_1$ 和 $T_{i - j + 1}$ 对齐),用
\begin{itemize}
\item $T_{i - j + 1..i-1} = P_{1..j-1}$
\item $T_i \ne P_j$
\end{itemize}
这两条信息找 $\set{i - j + 2, \dots, i}$ 里下一个可能的匹配点。

对于 $k = i - j + 2, \dots, i$,$k$ 是匹配点的必要条件有
\begin{itemize}
\item $T_{k..i-1}$ 是 $P$ 的一个前缀,即要有 
$$T_{k..i-1} = P_{1..i-k}$$
又由于 $T_{i - j + 1 ..i - 1} = P_{1..j-1}$ 所以 $T_{k..i-1} = P_{j+k-i..j-1}$
合起来就是
$$ P_{1..i-k} = P_{j+k-i..j - 1}$$ 
即 $P_{1..i-k}$ 得是 $P_{1..j-1}$ 的一个后缀。
\item $P_{i - k + 1} \ne P_j$
\end{itemize}
特别地,当 $k=i$ 时,$P_{1..i-k}$ 即 $P_{1..0}$,$T_{k..i-1}$ 即 $T_{i..i-1}$,二者都是空串,自然相等。此时只要有
$P_{i-k+1} \ne P_j$ 即 $P_1 \ne P_j$。
\end{frame}


\begin{frame}
  \frametitle{下一个匹配点}
  \begin{tcolorbox}
    \begin{itemize}
      \item $P_{1..i-k} = P_{j - i+k ..j - 1}$
      \item $P_{i - k + 1} \ne P_j$
      \end{itemize}
      \tcblower
    当 $T_i$ 和 $P_j$ 失配时,$i - j + 2, \dots, i$ 里下一个可能的匹配点就是满足上述两条件的最小的 $k$;若这样的 $k$ 不存在,则下一个可能的匹配点是 $i+1$。
  
  \end{tcolorbox}

\end{frame}

\begin{frame}{避免不必要的比较}
  \begin{tcolorbox}[title=要点]
    \begin{itemize}
      \item $P_{1..i-k} = P_{j - i+k ..j - 1}$
      \item $P_{i - k + 1} \ne P_j$
      \end{itemize}
      \tcblower
      当 $T_i$ 和 $P_j$ 失配时,
    \begin{itemize}
      \item 用来找下一个可能的匹配点的条件都是关于模式串$P$的。换言之,下一个可能的匹配点只与 $P$ 有关,所以可以在开始匹配之前算出来。
      \item {\kaishu 找到下一个可能的匹配点 $k$ 之后就从 $T_i$ 和 $P_{i - k + 1}$ 开始继续比较,不必从 $T_k$ 和 $P_1$ 开始比较。}
      \item 实际上我们要找的不是下一个匹配点 $k$ 而是 $i-k+1$,也就是下一次该拿 $T_i$ 和 $P$ 里的哪一个字符比较。
      \end{itemize}
  \end{tcolorbox}


\end{frame}

\begin{frame}
  \frametitle{避免不必要的比较}

  经过上面的分析,我们确实可以做到了避免不必要的比较:

  \begin{tcolorbox}
    一旦确定了 $T_i$ 等于 $P$ 里的某个字符 $P_j$,就再也不会(也不需要)将 $T_i$ 和 $P$ 的任何字符做比较。
  \end{tcolorbox}

\end{frame}



\begin{frame}[fragile]
  \frametitle{找到一个匹配时怎么办}

  此前我们讨论的结果是,当 $T_i$ 和 $P_j$ 失配时,下一次应该拿 $T_i$ 和 $P$ 里的哪一个字符比较。
  
  如果在比较的过程中,从某次失配往后直到 $T_i$ 和 $P_m$ 都没有发生失配,也就是说我们在 $T_i$ 处找到了一个 $P$ 时,接下来该做什么?


  \begin{tverb}[frame=bottomline, fontsize=\small]
               abcabcacab
\gray{babcbabcabcaabc}\lime{abca}\green{bcacab}c
\end{tverb}

\begin{tcolorbox}
  应该拿 $T_{i+1}$ 和 $P$ 里的某个字符比较。
\end{tcolorbox}


\begin{tverb}[frame=bottomline, fontsize=\small]
                       abcabcacab
\gray{babcbabcabcaabca}\pink{bcabcac}\lime{ab}c
\end{tverb}


沿用上面的分析方法,不难得出,应该拿 $T_{i+1}$ 和 $P_{j+1}$ 比较,这里 $j$ 是满足 $0 \le j < m$ 且 $P_{1..j} = P_{m-j+1,m}$ 的最大的 $j$。

\end{frame}



\begin{frame}
\frametitle{改进暴力匹配算法}
根据上面的分析,充分利用模式串 $P$,在开始匹配之前算出一个函数$\fail$。

\begin{itemize}
\item $\fail$ 的定义域是 $\set{1, 2, \dots, m, m + 1}$;陪域是 $\set{0, 1, 2, \dots, m}$。
\item $\fail(j)$:匹配过程中遇到$T_i$ 和 $P_j$ 失配时,接着拿 $T_i$ 和 $P_{\text{fail}(j)}$ 比较。($1 \le j \le m$)
\item $\fail(m+1)$:当在 $T_i$ 处找到一个匹配时,接着拿 $T_{i+1}$ 和 $P_{\fail(m+1)}$ 比较。
\item $\text{fail}(j) = 0$ 意味着下一个可能的匹配点是 $i+1$。
\item $\fail$ 函数完全由模式串 $P$ 决定。
\end{itemize}

$P=\str{abcabcacab}$ 对应的 $\fail$ 函数是

\begin{table}[]
\begin{tabular}{llllllllllll}
下标$j$ & 1  & 2  & 3 & 4 & 5 & 6 & 7 & 8 & 9 & 10 & 11\\
$S_j$ & \str{a} & \str{b} & \str{c} & \str{a} & \str{b} & \str{c} & \str{a} & \str{c} & \str{a}  & \str{b} \\
fail($j$) & 0  & 1  & 1 & 0 & 1 & 1 & 0 & 5  & 0  & 1 & 3
\end{tabular}
\end{table}

\end{frame}




\begin{frame}[fragile]{用$\fail$函数改进暴力匹配算法}

\begin{table}[]
\begin{tabular}{llllllllllll}
下标$j$ & 1  & 2  & 3 & 4 & 5 & 6 & 7 & 8 & 9 & 10 & 11 \\
%$S_j$ & \str{a} & \str{b} & \str{c} & \str{a} & \str{b} & \str{c} & \str{a} & \str{c} & \str{a}  & \str{b} \\
fail($j$) & 0  & 1  & 1 & 0 & 1 & 1 & 0 & 5  & 0  & 1 & 3
\end{tabular}
\end{table}

%minipage #1
\begin{minipage}[t]{0.33\textwidth}

\begin{tverb}[fontsize=\tiny]
abcabcacab \(\scriptstyle\fail(1)=0\)
\ured{b}abcbabcabcaabcabcabcacabc
\end{tverb}

\begin{tverb}[fontsize=\tiny]
 abcabcacab
\ul{\gray{b}}abcbabcabcaabcabcabcacabc
\end{tverb}

\begin{tverb}[fontsize=\tiny]
 abcabcacab
\gray{b}\ugrn{a}bcbabcabcaabcabcabcacabc
\end{tverb}

\begin{tverb}[fontsize=\tiny]
 abcabcacab
\gray{b}\green{a}\ugrn{b}cbabcabcaabcabcabcacabc
\end{tverb}

\begin{tverb}[fontsize=\tiny]
 abcabcacab
\gray{b}\green{ab}\ugrn{c}babcabcaabcabcabcacabc
\end{tverb}

\begin{tverb}[fontsize=\tiny]
 abcabcacab \(\scriptstyle\fail(4)=0\)
\gray{b}\green{abc}\ul{\red{b}}abcabcaabcabcabcacabc
\end{tverb}

\begin{tverb}[fontsize=\tiny]
     abcabcacab
\gray{babc}\ul{\gray{b}}abcabcaabcabcabcacabc
\end{tverb}

\begin{tverb}[fontsize=\tiny]
     abcabcacab
\gray{babcb}\ugrn{a}bcabcaabcabcabcacabc
\end{tverb}

\begin{tverb}[fontsize=\tiny]
     abcabcacab
\gray{babcb}\green{a}\ugrn{b}cabcaabcabcabcacabc
\end{tverb}

\begin{tverb}[fontsize=\tiny]
     abcabcacab
\gray{babcb}\green{ab}\ugrn{c}abcaabcabcabcacabc
\end{tverb}

\begin{tverb}[fontsize=\tiny]
     abcabcacab
\gray{babcb}\green{abc}\ugrn{a}bcaabcabcabcacabc
\end{tverb}

\begin{tverb}[fontsize=\tiny]
     abcabcacab
\gray{babcb}\green{abca}\underline{\green{b}}caabcabcabcacabc
\end{tverb}

\begin{tverb}[fontsize=\tiny]
     abcabcacab
\gray{babcb}\green{abcab}\underline{\green{c}}aabcabcabcacabc
\end{tverb}

\end{minipage}%
%
% minipage #2
\begin{minipage}[t]{0.33\textwidth}

\begin{tverb}[fontsize=\tiny]
     abcabcacab
\gray{babcb}\green{abcabc}\underline{\green{a}}abcabcabcacabc
\end{tverb}

\begin{tverb}[fontsize=\tiny]
     abcabcacab \(\scriptstyle\fail(8)=5\)
\gray{babcb}\green{abcabca}\ured{a}bcabcabcacabc
\end{tverb}

\begin{tverb}[fontsize=\tiny]
        abcabcacab \(\scriptstyle\fail(5)=1\)
\gray{babcbabc}\lime{abca}\ured{a}bcabcabcacabc
\end{tverb}

\begin{tverb}[fontsize=\tiny]
            abcabcacab
\gray{babcbabcabca}\ugrn{a}bcabcabcacabc
\end{tverb}

\begin{tverb}[fontsize=\tiny]
            abcabcacab
\gray{babcbabcabca}\green{a}\ugrn{b}cabcabcacabc
\end{tverb}

\begin{tverb}[fontsize=\tiny]
            abcabcacab
\gray{babcbabcabca}\green{ab}\ugrn{c}abcabcacabc
\end{tverb}

\begin{tverb}[fontsize=\tiny]
            abcabcacab
\gray{babcbabcabca}\green{abc}\ugrn{a}bcabcacabc
\end{tverb}

\begin{tverb}[fontsize=\tiny]
            abcabcacab
\gray{babcbabcabca}\green{abca}\ugrn{b}cabcacabc
\end{tverb}

\begin{tverb}[fontsize=\tiny]
            abcabcacab
\gray{babcbabcabca}\green{abcab}\ugrn{c}abcacabc
\end{tverb}

\begin{tverb}[fontsize=\tiny]
            abcabcacab
\gray{babcbabcabca}\green{abcabc}\ugrn{a}bcacabc
\end{tverb}

\begin{tverb}[fontsize=\tiny]
            abcabcacab \(\scriptstyle\fail(8)=5\)
\gray{babcbabcabca}\green{abcabca}\ured{b}cacabc
\end{tverb}

\begin{tverb}[fontsize=\tiny]
               abcabcacab
\gray{babcbabcabcaabc}\lime{abca}\ugrn{b}cacabc
\end{tverb}

\end{minipage}%
%
%minipage #3
\begin{minipage}[t]{.33\textwidth}

\begin{tverb}[fontsize=\tiny]
               abcabcacab
\gray{babcbabcabcaabc}\lime{abca}\green{b}\ugrn{c}acabc
\end{tverb}

\begin{tverb}[fontsize=\tiny]
               abcabcacab
\gray{babcbabcabcaabc}\lime{abca}\green{bc}\ugrn{a}cabc
\end{tverb}

\begin{tverb}[fontsize=\tiny]
               abcabcacab
\gray{babcbabcabcaabc}\lime{abca}\green{bca}\ugrn{c}abc
\end{tverb}

\begin{tverb}[fontsize=\tiny]
               abcabcacab
\gray{babcbabcabcaabc}\lime{abca}\green{bcac}\ugrn{a}bc
\end{tverb}

\begin{tverb}[fontsize=\tiny]
               abcabcacab
\gray{babcbabcabcaabc}\lime{abca}\green{bcaca}\ugrn{b}c
\end{tverb}

\begin{tverb}[fontsize=\tiny]
                       abcabcacab
\gray{babcbabcabcaabcabcabcac}\lime{ab}\ugrn{c}
\end{tverb}
\end{minipage}

\end{frame}


\begin{frame}{用$\fail$函数改进暴力匹配算法}

观察上述过程,可以看出

\begin{itemize}
    \item 算法的过程大概是把$T_1$ 和 $P$里若干字符比较,把$T_2$ 和 $P$ 里若干字符比较,……,把$T_n$ 和 $P$ 里若干字符比较。
    \item 对于$T_i$的比较完成后,我们知道既是 $T_{1..i}$ 的后缀又是 $P$ 的前缀的字符串最长能有多长。
    \item 每次比较之后要么文本串$T$里的下划线右移,要么模式串$P$右移。二者的移动次数都不会超过$n$。
    \item 算出 $\fail$ 函数之后可以在 $O(n)$ 的时间内完成字符串匹配。
\end{itemize}

\tcbox{如何计算$\fail$函数?}
\end{frame}

\begin{frame}{要点}
  上面叙述的在文本串 $T$ 中寻找(或者说“搜索”、“匹配”)模式串 $P$ 的过程,从另一个角度看就是
  \begin{tcolorbox}
    对于文本串 $T$ 的每个前缀 $T_{1..i}$,计算既是 $T_{1..i}$ 的后缀又是文本 $P$ 的前缀的字符串最长有多长。
    
    若这样的字符串的最大长度是 $|P|$,那么 $i-|P| + 1$ 就是一个匹配点。
  \end{tcolorbox}
\end{frame}



\begin{frame}[fragile]
  \frametitle{用 fail 函数进行匹配}

  \begin{tcolorbox}
    \begin{cverb}[fontsize=\normalsize]
  int j = 1;
  for (int i = 1; i <= n; i++) {
    while (j != 0 and P[j] != T[i])
      j = fail[j];
    j++;
    if (j == m + 1) {
      //i - m + 1 是匹配点;
      j = fail[j];
    }
  }
    \end{cverb}
    \end{tcolorbox}

\end{frame}





\begin{frame}{真前缀,前后缀}

\begin{block}{真前缀}
  对于非空字符串 $S$,设 $S$ 的长度是 $n$。我们把 $S$ 的长度\emph{小于} $n$ 的前缀称为 $S$ 的{\kaishu 真前缀}(proper prefix,也可译作``严格前缀'')。

  例如,$S= \str{abc}$ 的真前缀有:空串,\str{a},\str{ab}。
\end{block}

\begin{block}{前后缀}
对于非空字符串 $S$,若字符串 $T$ 既是 $S$ 的真前缀也是 $S$ 的后缀,则称 $T$ 为 $S$ 的前后缀。例如 
\begin{itemize}
  \item \str{abc} 的前后缀有:空串;
  \item \str{ababa} 的前后缀有:空串,\str{a},\str{aba};
  \item \str{aaa} 的前后缀有:空串,\str{a},\str{aa}。
\end{itemize}
\tcbox{任何非空字符串都有前后缀。}
\end{block}

“前后缀”英文叫做“border”。
\end{frame}

\begin{frame}
  \frametitle{fail 函数和前后缀的关系}

  $\fail(j)$ 就是满足
  \begin{itemize}
    \item $P_{1..k-1}$ 是$P_{1..j-1}$ 的前后缀,且
    \item $P_{k} \ne P_j$
  \end{itemize}
  这两个条件的{\kaishu 最大的} $k$。

  \begin{block}{}
    换一种说法,$\fail(j)$ 就是 $b = 0, \dots, j-2$ 中,满足 $P_{1..b}$ 是 $P_{1..j-1}$ 的{\kaishu 后缀},且 $P_{b+1} \ne P_j$ 的最大的 $b$ 再加 $1$。
  \end{block}
\end{frame}



\begin{frame}
\frametitle{计算fail函数}

\begin{block}{}
\begin{itemize}
  \item 边界条件:$\text{fail}(1) = 0$。
  \item 对于 $j = 2, 3, \dots, m$,若 $1, \dots, j - 1$ 中存在 $k$ 满足
  {
  \begin{itemize}
  \item $P_{1..k - 1} = P_{j - (k - 1) .. j - 1}$ 且 $P_{k} \ne P_j$。
  \end{itemize}
  则 $\fail(j) = $满足此条件的最大的 $k$,否则 $\fail(j) = 0$。
  }
  \end{itemize}
\end{block}





对于 $j = 2, \dots, m$,当 $T_i$和 $P_j$ 失配时,我们做两件事:

\begin{itemize}
    \item 先用 $P_{1..k - 1} = P_{j - (k - 1) .. j - 1}$ 这个条件把
          模式串尽可能少地往右移动一些。也就是说,
          我们先从 $0, \dots, j - 2$ 中找到一个最大的整数 $t$ 满足 $P_{1..t} = P_{j-t .. j-1}$;即找到 $P_{1..j-1}$ 的{\kaishu 最长前后缀}。
          把 $P$ 向右移动一些,使 $P_{t}$ 和 $T_{i - 1}$ 对齐。(向右移动 $j - 1 - t$格,$P_1$ 和 $T_{i - t}$ 对齐)。 % P_l 和 T_{i-1} 对齐,P_1 和
% j --
    \item 然后看 $P_{t +1}$ 是否等于 $P_j$,
    \begin{itemize}
      \item  若 $P_{t + 1} \ne P_{j}$,则 $T_{i}$ 有可能等于 $P_{t + 1}$( $i - t$ 可能是匹配点),
      接着从 $T_i$ 和 $P_{t +1}$ 继续比较,于是 $\fail(j) = t + 1$。
      \item  否则 $P_{t+1} = P_j$, $T_{i}$ 一定不等于 $P_{t + 1}$,换言之,从 $T_{i-t}$ 开始匹配会在 $P_{t + 1}$ 处失配,于是 $\fail(j) = \fail(t + 1)$。
    \end{itemize}
   
   
\end{itemize}

\end{frame}

\begin{frame}[fragile]{计算fail函数:例一,计算 $\fail(4)$}

\begin{tverb}
 abcabcacab
\gray{*}\green{abc}\red{x}{**********}
\end{tverb}

$P_{1..3}$ 的最长前后缀长度是 $0$,先把 $P$ 向右移动 $(3-0)$ 格,使得 $P_1$ 和 $T_5$ 对齐。
接下来要比较 \str{x} 和 $P_4$。

\begin{block}{}
\begin{tverb}
    abcabcacab
\gray{*abc}\red{x}{**********}
\end{tverb}
由 $P_4 \ne \str{x}$ 且 $P_1 = P_4$ 可知 $P_1 \ne \str{x}$。
当在 $P_4$ 处失配时,把 $P$ 右移三格,一定会在 $P_1$ 处失配,进而知 $\fail(4) = \fail(1) = 0$。
\end{block}



\begin{tabular}{p{.5\textwidth} p{.5\textwidth}}
\begin{tverb}
 abcabcacab
\gray{*}\green{abc}\red{x}{**********}
\end{tverb}
 &
\begin{tverb}
     abcabcacab
\gray{*abcx}{**********}
\end{tverb}

\end{tabular}
\end{frame}


\begin{frame}[fragile]{计算fail函数:例二,计算$\fail(8)$}

\begin{tverb}
 abcabcacab
\gray{*}\green{abcabca}\red{x}******
\end{tverb}

在 $P_8$ 处失配,$P_{1..7}$(\str{abcabca}) 的最长前后缀(\str{abca})的长度是 $4$,先把 $P$ 右移 $(7 - 4)$ 格。接下来要比较 \str{x} 和 $P_5$。

\begin{tverb}
    abcabcacab
\gray{*}\green{abcabca}\red{x}******
\end{tverb}

我们知道 $\str{x} \ne P_8$, 但是 $P_5 \ne P_8$,所以不能断定 \str{x} 和 $P_5$ 是否相等。换言之,当文本串里的字符 \str{x} 和 $P_8$ 失配时,接下来要比较 \str{x} 和 $P_5$,于是 $\fail(8) = 5$。

\begin{tabular}{p{.5\textwidth} p{.5\textwidth}}
\begin{tverb}
 abcabcacab
\gray{*}\green{abcabca}\red{x}******
\end{tverb}
 &
\begin{tverb}
    abcabcacab
\gray{*abc}\lime{abca}x******
\end{tverb}

\end{tabular}

\end{frame}

\begin{frame}{计算 $\fail$ 函数}

我们得到了递推计算 $\fail$ 函数的方法。

对于 $j = 1, \dots, m$,把 $P_{1..j}$ 的最长前后缀的长度记作 $\pi(j)$。
\begin{itemize}
    \item 边界条件:$\fail(1) = 0, \pi(1) = 0$。
    \item $\pi(1) \to \fail(2)$
    \item $\pi(2) \to \fail(3)$
    \item \dots
\end{itemize}

现在我们要考虑的问题是

\begin{tcolorbox}
  算出 $\pi(1), \dots, \pi(j - 1), \fail(1), \dots, \fail(j)$ 之后,如何计算 $\pi(j)$?
\end{tcolorbox}
\end{frame}


\begin{frame}[fragile]{计算 $P_{1..j}$ 的最长前后缀}

  回顾上面所说的看待在文本串 $T$ 中寻找模式串 $P$ 的另一个角度:
  
  \begin{tcolorbox}
    在文本串 $T$ 中寻找模式串 $P$,实际上做的事情是:对于 $T$ 的每个前缀 $T_{1..i}$,计算既是 $T_{1..i}$ 的后缀又是 $P$ 的前缀的字符串最长有多长。
  \end{tcolorbox}
  
  反过来,
  \begin{tcolorbox}
    在文本串 $P_{2..m}$ 寻找模式串$P_{1..m-1}$ 的过程中,可以把每个 $P_{1..j}$ 的最长前后缀的长度算出来。
    \tcblower
\begin{tverb}
 abcabcaca\lightgray{b}   (模式串)
\lightgray{a}bcabcacab    (文本串)
\end{tverb}
  \end{tcolorbox}
\end{frame}


\begin{frame}[fragile]{计算 $P_{1..j}$ 的最长前后缀} %TODO:此页需要修改
$P = \str{abcabcacab}$。
    计算$P_{1..5} = \str{abcab}$的最长前后缀的长度,即 $\pi(5)$。

此时我们知道 $\pi(4) = 1$,这个信息可以用下图来表现。
\begin{tverb}
   abcabcacab
abc\lime{a}bcacab
\end{tverb}
我们知道 $\pi(5)$ 不超过 $2$。看 $\pi(5)$ 是否等于 $2$,即比较$P_5$ 和 $P_2$,结果是 $P_5 = P_2$
\begin{tverb}
   abcabcacab
abc\lime{a}\green{b}cacab
\end{tverb}
于是知 $\pi(5)=2$。同理知 $\pi(6) = 3, \pi(7) = 4$。

\begin{tverb}
   abcabcacab
abc\lime{a}\green{bca}cab
\end{tverb}

现在来计算 $\pi(8)$,我们知道 $\pi(8) \le 5$,看 $\pi(8)$ 是否等于 $5$,即比较 $P_8$ 和 $P_5$,结果是 $P_8 \ne P_5$
\begin{tverb}
   abcabcacab
abc\lime{a}\green{bca}\red{c}ab
\end{tverb}
此时的情形相当于在文本串 $P$ 中匹配模式串 $P$,文本串的 $P_8$ 和模式串的 $P_5$ 失配了。
\end{frame}




\begin{frame}[fragile]{代码:计算 fail 函数}
\begin{tcolorbox}
\begin{cverb}[fontsize=\normalsize]
fail[1] = 0;
int t = 1;
for (int j = 2; j <= m; j++) {
// t是P[1..j - 1]的最长前后缀的长度加一
  if (P[j] != P[t])
    fail[j] = t;
  else
    fail[j] = fail[t];
  // 计算P[1..j]的最长前后缀
  while (t != 0 and P[t] != P[j])
    t = fail[t];
  ++t;
}
//此时t是P[1..m]的最长前后缀的长度加一。
fail[m + 1] = t;
\end{cverb}
\end{tcolorbox}

\end{frame}


\begin{frame}[fragile]{代码}
  \begin{tcolorbox}
  \begin{cverb}
  fail[1] = 0;
  int t = 1;
  for (int j = 2; j <= m; j++) {
  // t是P[1..j - 1]的最长前后缀的长度加一
    if (P[j] != P[t])
      fail[j] = t;
    else
      fail[j] = fail[t];
    while (t != 0 and P[t] != P[j])
      t = fail[t];
    ++t;
  }
  //此时t是P[1..m]的最长前后缀的长度加一。
  fail[m + 1] = t;
  int j = 1;
  for (int i = 1; i <= n; i++) {
    while (j != 0 and P[j] != T[i])
      j = fail[j];
    j++;
    if (j == m + 1) {
      //i - m + 1 是匹配点;
      j = fail[j];
    }
  }
  \end{cverb}
  \end{tcolorbox}
  
  \end{frame}

\begin{frame}[fragile]{代码}
字符串下标从$0$开始实际上更为常用,为此只需将上面的代码稍作改动。

\begin{tcolorbox}[sidebyside]
\begin{cverb}
fail[1] = 0;
int t = 1;
for (int j = 2; j <= m; j++) {
// t是P[1..j - 1]的最长前后缀的长度加一
  if (P[j] != P[t])
    fail[j] = t;
  else
    fail[j] = fail[t];
  while (t != 0 and P[t] != P[j])
    t = fail[t];
  ++t;
}
//此时t是P[1..m]的最长前后缀的长度加一。
fail[m + 1] = t;
int j = 1;
for (int i = 1; i <= n; i++) {
  while (j != 0 and P[j] != T[i])
    j = fail[j];
  j++;
  if (j == m + 1) {
    //i - m + 1 是匹配点;
    j = fail[j];
  }
}
\end{cverb}
\tcblower
\begin{cverb}[fontsize=\tiny]
fail[0] = -1;
int t = 0;
for (int j = 1; j < m; j++) {
// t 是 P[0..j-1]的最长前后缀的长度
  if (P[j] != P[t])
    fail[j] = t;
  else
    fail[j] = fail[t];
  while (t != 0 and P[t] != P[j])
    t = fail[t];
  ++t;
}
//此时t是P[0..m-1]的最长前后缀的长度。
fail[m] = t;
int j = 0;
for (int i = 0; i < n; i++) {
  while (j != -1 and P[j] != T[i])
    j = fail[j];
  j++;
  if (j == m) {
    //i-m+1是匹配点。
    j = fail[j];
  }
}
\end{cverb}
\end{tcolorbox}

\end{frame}


\begin{frame}[fragile]{例题:abc257g Prefix Concatenation(前缀拼接)}
给你两个由小写英文字母构成的字符串 $S$,$T$。

求满足下列条件的最小正整数 $k$。
\begin{itemize}
    \item $T$可以由 $k$ 个 $S$ 的前缀拼接得到。
\end{itemize}
若$T$不能由若干$S$的前缀拼接得到,输出$-1$。

\begin{itemize}
    \item $1 \le |S|, |T| \le \num{5e5}$
\end{itemize}

\begin{tcolorbox}[sidebyside,title=样例]
\begin{cverb}
aba
ababaab
\end{cverb}
\tcblower
\begin{cverb}
3
\end{cverb}
\end{tcolorbox}
\verb|ababaab = ab + aba + ab|
\end{frame}



\end{document}
%