%! Author = zjs
%! Date = 6/27/22

% Preamble
\documentclass{ctexbeamer}
\usepackage{braket}
\usepackage{fancyvrb}
%\usepackage[outputDir=../out, hybrid]{markdown}
\usepackage{textcomp}
\usepackage{amsmath}

\newcommand{\red}[1]{\textcolor{red}{#1}}
\newcommand{\green}[1]{\textcolor{green}{#1}}
\newcommand{\gray}[1]{\textcolor{gray}{#1}}
\newcommand{\lime}[1]{\textcolor{lime}{#1}}
\newcommand{\purple}[1]{\textcolor{purple}{#1}}
\newcommand{\str}[1]{\texttt{#1}}
\newcommand{\fail}{\mathsf{fail}}

\DefineVerbatimEnvironment%tight verbatim
{tverb}{Verbatim}
{commandchars=\\\{\},baselinestretch=0,frame=bottomline, fontsize=\small}

\newenvironment{fframe}{\begin{frame}[fragile]}{\end{frame}}


\DefineVerbatimEnvironment%tight verbatim
{bverb}{BVerbatim}
{commandchars=\\\{\},baselinestretch=0, fontsize=\small}
% Packages

% Document
\begin{document}
\title{KMP算法}
\author{邹家树}
\institute{小码王}
\maketitle

\begin{frame}[fragile]
	\frametitle{问题:字符串匹配}

\begin{itemize}
\item 字符串下标从1开始。
\item 字符串 $S$ 的第 $i$ 个字符记作 $S_i$。
\end{itemize}

    模式串$P$,文本串$T$。
    模式串在文本串的哪些地方出现过?
    $P=\str{aba}$,$T=\str{ababa}$。答案:$\set{1, 3}$。

\begin{itemize}
\item 文本串长度记作 $n$
\item 模式串长度记作 $m$
\end{itemize}

若 $T_{i..m+i-1} = P$ 则称 $i$ 是\emph{匹配点}。

\end{frame}


\begin{frame}[fragile]
	\frametitle{暴力匹配算法}

\begin{itemize}
\item 第$i$次匹配:把模式串开头和文本串第$i$个字符对齐,从头开始比较。遇到文本串和模式串里对应字符不相同(称为失配)就停止。
\item 共匹配 $n-m+1$次。
\end{itemize}
\end{frame}

\begin{frame}[fragile]
\frametitle{暴力匹配算法示例}
P= \texttt{abcabcacab},T=\texttt{babcbabcabcaabcabcabcacabc}。

\begin{minipage}[t]{0.5\textwidth}
	\begin{tverb}
\red{a}bcabcacab \(1\)
babcbabcabcaabcabcabcacabc
	\end{tverb}

	\begin{tverb}
 \green{abc}\red{a}bcacab \(4\)
babcbabcabcaabcabcabcacabc
	\end{tverb}


	\begin{tverb}
  \red{a}bcabcacab \(1\)
babcbabcabcaabcabcabcacabc
	\end{tverb}

	\begin{tverb}
   \red{a}bcabcacab \(1\)
babcbabcabcaabcabcabcacabc
	\end{tverb}

	\begin{tverb}
    \red{a}bcabcacab \(1\)
babcbabcabcaabcabcabcacabc
	\end{tverb}

	\begin{tverb}
     \green{abcabca}\red{c}ab \(8\)
babcbabcabcaabcabcabcacabc
	\end{tverb}

	\begin{tverb}
      \red{a}bcabcacab \(1\)
babcbabcabcaabcabcabcacabc
	\end{tverb}

	\begin{tverb}
       \red{a}bcabcacab \(1\)
babcbabcabcaabcabcabcacabc
	\end{tverb}

	\begin{tverb}
        \green{abca}\red{b}cacab \(5\)
babcbabcabcaabcabcabcacabc
	\end{tverb}

\end{minipage}%
\begin{minipage}[t]{.5\textwidth}

\begin{tverb}
         \red{a}bcabcacab \(1\)
babcbabcabcaabcabcabcacabc
	\end{tverb}

\begin{tverb}
          \red{a}bcabcacab \(1\)
babcbabcabcaabcabcabcacabc
	\end{tverb}

\begin{tverb}
           \green{a}\red{b}cabcacab \(2\)
babcbabcabcaabcabcabcacabc
    	\end{tverb}

\begin{tverb}
            \green{abcabca}\red{c}ab \(7\)
babcbabcabcaabcabcabcacabc
            	\end{tverb}

\begin{tverb}
             \red{a}bcabcacab \(1\)
babcbabcabcaabcabcabcacabc
\end{tverb}

\begin{tverb}
              \red{a}bcabcacab \(1\)
babcbabcabcaabcabcabcacabc
            	\end{tverb}

\begin{tverb}
               \green{abcabcacab} \(10\)
babcbabcabcaabcabcabcacabc
\end{tverb}

\begin{tverb}
                \red{a}bcabcacab \(1\)
babcbabcabcaabcabcabcacabc
            	\end{tverb}


\end{minipage}



一共比较了 $47$ 次。

\end{frame}

\begin{frame}
	\frametitle{暴力匹配算法的时间复杂度}

最坏情况时间复杂度是 $O(nm)$。


P=\texttt{aaaaaaab},T=\texttt{aaaaaaaaaaaaaaaaaaaab}。
\end{frame}

\begin{frame}[fragile]
\frametitle{改进暴力匹配算法}

观察暴力匹配算法前五次匹配过程。

	\begin{tverb}
\red{a}bcabcacab \(1\)
babcbabcabcaabcabcabcacabc
	\end{tverb}

\begin{tverb}
 abcabcacab \(4\)
b\green{abc}\red{b}abcabcaabcabcabcacabc
	\end{tverb}


	\begin{tverb}
  \red{a}bcabcacab \(1\)
babcbabcabcaabcabcabcacabc
	\end{tverb}

	\begin{tverb}
   \red{a}bcabcacab \(1\)
babcbabcabcaabcabcabcacabc
	\end{tverb}

	\begin{tverb}
    \red{a}bcabcacab \(1\)
babcbabcabcaabcabcabcacabc
	\end{tverb}

从``第二次失配发生在$P_4$处'',我们可以知道关于文本串$T$的两件事:
\begin{itemize}
\item $T$ 的第二到第四个字符等于 $P$ 的前三个字符(\str{abc})。
\item $T_5$不等于$P_4$(\str{a})。
\end{itemize}
用这两个条件就能断定随后的三次匹配在$P_1$处就失配。

充分挖模式串的特点,失配信息可以用来预判之后的匹配结果。

\end{frame}

\begin{frame}[fragile]
\frametitle{改进暴力匹配算法}

\begin{tverb}
abcabcacab
\red{b}abcbabcabcaabcabcabcacabc \(1\)
\end{tverb}

	\begin{tverb}
 abcabcacab
b\green{abc}\red{b}abcabcaabcabcabcacabc \(4\)
	\end{tverb}

\begin{tverb}
     abcabcacab
ba\gray{bcb}abcabcaabcabcabcacabc
\end{tverb}

\begin{tverb}
     abcabcacab
babcb\green{abcabca}\red{a}bcabcabcacabc \(8\)
\end{tverb}

\begin{tverb}
        abcabcacab
babcbabc\lime{abca}\purple{a}bcabcabcacabc
\end{tverb}

\begin{tverb}
        abcabcacab
babcbabc\lime{abca}\red{a}bcabcabcacabc \(1\)
\end{tverb}

\begin{tverb}
            abcabcacab
babcbabca\gray{bca}\purple{a}bcabcabcacabc
\end{tverb}

\begin{tverb}
            abcabcacab
babcbabcabca\green{abcabca}\red{b}cacabc \(8\)
\end{tverb}

\begin{tverb}
               abcabcacab
babcbabcabcaa\gray{bc}\lime{abca}\purple{b}cacabc
\end{tverb}

\begin{tverb}
               abcabcacab
babcbabcabcaabc\lime{abca}\green{bcacab}c \(6\)
\end{tverb}

\begin{tverb}
                       abcabcacab
babcbabcabcaabca\gray{bcabcac}\lime{ab}c           一共比较了\(28\)次
\end{tverb}

\end{frame}

\begin{frame}[fragile]
\frametitle{改进暴力匹配算法}


当 $T_i$ 和 $P_j$ 失配时(此时 $P_1$ 和 $T_{i - j + 1}$对齐),用
\begin{itemize}
\item $P_{1..j-1} = T_{i - j + 1..i-1}$
\item $P_j \ne T_i$
\end{itemize}
两条件找$\set{i - j + 2, \dots, i}$里下一个可能的匹配点。

\begin{block}{}
对于 $k = i - j + 2, \dots, i - 1$,$k$ 是匹配点的必要条件有
\begin{itemize}
\item $P_{1..i-k} = T_{k..i-1} $ 即 $P_{1..i-k} = P_{j - i+k ..j - 1}$
\item $P_{i - k + 1} \ne P_j$
\end{itemize}

$i$ 是匹配点的必要条件有
\begin{itemize}
\item $P_1 \ne P_j$
\end{itemize}

\end{block}


\begin{block}{要点}
\begin{itemize}
\item 用来找下一个可能的匹配点的条件都是关于模式串$P$的。
\item 找到下一个可能的匹配点 $k$ 之后就从 $T_i$ 和 $P_{i - k + 1}$ 开始继续比较,不必从 $T_k$ 和 $P_1$ 开始比较。
\end{itemize}
\end{block}



\end{frame}

\begin{frame}
\frametitle{改进暴力匹配算法}
充分利用模式串 $P$ 的特点,事先算出一个函数fail。

\begin{itemize}
\item $\text{fail}(j)$ := 匹配过程中遇到$T_i$ 和 $P_j$ 失配时,接着拿 $T_i$ 和 $P_{\text{fail}(j)}$ 比较。
\item $\text{fail}(j) = 0$ 意味着下一个可能的匹配点是 $i+1$。
\end{itemize}




%abcabcacab
%         abcab
\begin{table}[]
\begin{tabular}{llllllllllll}
下标$j$ & 1  & 2  & 3 & 4 & 5 & 6 & 7 & 8 & 9 & 10 \\
$S_j$ & \str{a} & \str{b} & \str{c} & \str{a} & \str{b} & \str{c} & \str{a} & \str{c} & \str{a}  & \str{b} \\
fail($j$) & 0  & 1  & 1 & 0 & 1 & 1 & 0 & 5  & 0  & 1
\end{tabular}
\end{table}

\begin{block}{}
如何计算 fail 函数?
\end{block}




\end{frame}

\begin{frame}[fragile]
\frametitle{计算fail函数}


\begin{itemize}
\item 边界条件:$\text{fail}(1) = 0$。
\item 对于 $j = 2, 3, \dots, m$,若 $1, \dots, j - 1$ 中存在 $k$ 满足
{
\begin{itemize}
\item $P_{1..k - 1} = P_{j - (k - 1) .. j - 1}$ 且 $P_{k} \ne P_j$。
\end{itemize}
则 $\fail(j) = $满足此条件的最大的 $k$,否则 $\fail(j) = 0$。
}
\end{itemize}

\begin{block}{}
上面写出的 $\fail(j) =  \dots$ 是依照``充分利用模式串 $P$ 的特点进行预判,从而避免不必要的比较''这个想法给 $\fail(j)$ 下的定义。不过从这个定义并不容易看出如何计算 $\fail(j)$。
\end{block}


\begin{block}{}
我们可以从另一个角度来思考 $\fail(j)$ 应该等于多少。
\end{block}

\end{frame}

\end{document}
%